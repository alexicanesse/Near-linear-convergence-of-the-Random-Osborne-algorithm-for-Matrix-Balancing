\documentclass[a4paper,dvipsnames, 11pt]{article}
\usepackage{latexsym}
\usepackage{amsmath}
%\usepackage{mathspec}
%\usepackage{amsthm}
% \usepackage{amssymb}
% \usepackage{textcomp} 
% \usepackage{ulem}
% % \usepackage{fontspec}     
\usepackage[british]{babel}
\usepackage{enumitem}
% \usepackage{pifont}
\usepackage[a4paper, margin=1.2in]{geometry}
\usepackage{hyperref}
\usepackage[dvipsnames, table]{xcolor}
\usepackage{tikz}
% \usepackage{mathrsfs}
\usepackage{cleveref}
\usepackage{caption}
\usepackage{listings}
% \usepackage{eucal}
\usepackage{dsfont}
% \usepackage[most]{tcolorbox}
% \usepackage{mathdots}
\usepackage{minted} %? used for code-integration
% \usepackage{titling}
\usepackage[bibitemsep=100pt, citestyle=alphabetic,bibstyle=alphabetic]{biblatex}

\definecolor{darkmagenta}{RGB}{99,0,99}
\hypersetup{
    colorlinks,
    citecolor=darkmagenta,
    filecolor=darkmagenta,
    linkcolor=darkmagenta,
    urlcolor=darkmagenta
} %? Change hyperlink's styles

\setlist[itemize,1]{label=\(\color{magenta}\bullet\)}

%%%%%%%%%%%%%%? courbes
\usepackage{amsfonts,amscd}
\usepackage{pgfplots}
%%%%%%%%%%%%%?

%%%%%%%%%%%%%
\DeclareMathOperator{\mat}{Mat}
\DeclareMathOperator{\card}{Card}
\DeclareMathOperator{\len}{len}
\DeclareMathOperator{\erf}{erf}
\DeclareMathOperator{\argmax}{argmax}
%%%%%%%%%%%%%


%%%%%%%%%%%%%
\newcommand{\dt}{\mathrm{d}}
\newcommand{\un}{\mathds1}
\renewcommand{\o}{\scriptstyle\mathcal{O}}
\renewcommand{\O}{\mathcal{O}}
\renewcommand{\emptyset}{\varnothing}
\newcommand{\lbd}{\lambda}
\renewcommand{\phi}{\varphi}
\def\inte#1 #2 { [\![#1,#2]\!] }
%%%%%%%%%%%%%


%%%%%%%%%%%%%
\newcommand{\N}{\mathbb{N}}
\newcommand{\D}{\mathbb{D}}
\newcommand{\R}{\mathbb{R}}
\newcommand{\Z}{\mathbb{Z}}
\newcommand{\C}{\mathbb{C}}
\newcommand{\K}{\mathbb{K}}
\newcommand{\Q}{\mathbb{Q}}
\renewcommand{\P}{\mathbb{P}}
\newcommand{\M}{\mathcal{M}}
\renewcommand{\r}{\mathcal{R}}
\renewcommand{\S}{\mathcal{S}}
%%%%%%%%%%%%%


\renewcommand{\arraystretch}{1.4} %? For better matrix
\renewcommand{\thesection}{\arabic{section}} %? Roman number for sections


\setcounter{secnumdepth}{4}
\setcounter{tocdepth}{4}


% Define theorem-like environments
\newtheorem{theor}{Theorem}[section]
\newtheorem{lemma}{Lemma}[section]
\newtheorem{definition}{Definition}[section]
\newtheorem{notation}{Notation}[section]
\newtheorem{remark}{Remark}[section]

% Custom commands for convenience
\newcommand{\thm}[1]{\begin{theor}#1\end{theor}}
\newcommand{\lem}[1]{\begin{lemma}#1\end{lemma}}
\newcommand{\defn}[1]{\begin{definition}#1\end{definition}}
\newcommand{\nota}[1]{\begin{notation}#1\end{notation}}
\newcommand{\rem}[1]{\begin{remark}#1\end{remark}}

% Use cleverref for enhanced referencing
\crefname{theor}{theorem}{theorems}
\crefname{lemma}{lemma}{lemmas}
\crefname{definition}{definition}{definitions}
\crefname{remark}{remark}{remarks}

\newenvironment{code}{
    \VerbatimEnvironment\begin{minted}[mathescape,
        linenos,
        numbersep=5pt,
        gobble=2,
        frame=lines,
        escapeinside=||,
        framesep=2mm]{python}%
}{
        \end{minted}%
}

\hbadness=10000
\vbadness=10000
\flushbottom
\addbibresource{rapport.bib}
\usepackage{tkz-base}
\usepackage{algorithm}
\usepackage{algorithmic}
\usepackage{wrapfig}
\usepackage{subcaption}
\usepackage{changepage}
\usepackage[]{tocbibind}
\usepackage[useregional]{datetime2}
\usepackage{adjustbox}
\usepackage{pdfpages}
\usepackage{tabularx}
\usepackage{array}
\usepackage{cleveref}
\usepackage{multicol}
\usepackage{graphicx}
\usepackage{multirow}
\usepackage[multiple]{footmisc}
\usepackage[minNoteWidth=1cm,
    colorinlistoftodos,
    backgroundcolor=magenta!30!white,
    bordercolor=magenta]{luatodonotes}
\usepackage{amssymb}

\newcolumntype{C}{>{\centering\arraybackslash}X}
\newcolumntype{s}{>{\hsize=.3\hsize\linewidth=\hsize}C}
\newcolumntype{D}{>{\hsize=.4\hsize\linewidth=\hsize}C} %Double width column

\setlength{\bibhang}{0pt}

\defbibenvironment{bibliography}
{\list{\printtext[labelalphawidth]{\printfield{labelprefix}\printfield{labelalpha}}}% Updated line
{\setlength{\labelwidth}{\labelalphawidth}% Updated line
\setlength{\leftmargin}{0pt}%
\setlength{\labelsep}{\biblabelsep}%
\setlength{\itemsep}{\bibitemsep}%
\setlength{\itemsep}{0pt}% Set to 0pt to remove vertical space
\setlength{\parsep}{\bibparsep}}%
\renewcommand*{\makelabel}[1]{\hss\hspace{\dimexpr\labelalphawidth+\labelsep}##1}}
{\endlist}
{\item}

% Redefine the footnotetext command to remove indentation
\makeatletter
\renewcommand\@makefntext[1]{%
    \noindent\makebox[0em][r]{\@thefnmark\hspace{5pt}}#1}
\makeatother

\newcommand{\reftodo}[1][]{\todo[backgroundcolor=green!30!white, bordercolor=green, #1]{\thesubsubsection{}~Ref} {\color{red} REF}}
\newcommand{\missingtext}[1][]{\todo[inline, caption=\thesubsection~Fill missing text, backgroundcolor=cyan!30!white, bordercolor=cyan]{#1}}
%\usepackage[]{todonotes}

\setlength\parindent{0pt}

\usetikzlibrary{shapes.multipart}

\makeatletter
\newenvironment{fixedfigure}
{\def\@captype{figure}\center}
{\endcenter}
\makeatother



% \usepackage{graphicx,txfonts}

\addto\captionsbritish{
    \renewcommand{\contentsname}%
    {Table of content}%
}

\renewcommand{\today}{\number\day\textsuperscript{\scriptsize\text{th}} \DTMenglishmonthname{\month} \number\year}


\makeatletter
\newcommand\setxveclength[5]{% newmacro, node1, node2
    \pgfpointdiff{#2}{#3}
    \edef#1{\the\pgf@x}
}
\makeatother

\sloppy                  

\pgfplotsset{compat=1.16}

\begin{document}


%\setmathfont{Latin Modern Math}
%\setmathfont[range={\mathscr,\mathbfscr}]{XITS Math}

% \maketitle



\begin{titlepage}
    \begin{center}
        \vspace*{1cm}
            
        \Huge
        \textbf{Random Osborne Algorithm for Matrix Balancing}
            
        \vspace{0.5cm}
        \LARGE
        Optimal transport report
            
        \vspace{2.5cm}
            
        \huge {\sc Alexi Canesse\footnote{\href{mailto:alexi.canesse@ens-lyon.fr}{alexi.canesse@ens-lyon.fr}, ENS de Lyon, France}\footnote{\href{mailto:alexi.canesse@ens-paris-saclay.fr}{alexi.canesse@ens-paris-saclay.fr}, ENS Paris-Saclay, France}}\\\Large
            
        \vfill

        Optimal transport course\\
        Part of the MVA program at ENS Paris-Saclay.
            
        \vspace{2cm}
        \includegraphics[width=0.5\textwidth]{./figures/logo.pdf}
        \vspace{2cm}

        \Large
        Computer Science and Mathematics\\
        École Normale Supérieur Paris-Saclay\\
        Orsay, France\\
        \today
            
    \end{center}
\end{titlepage}





\newpage

%\tableofcontents
\makeatletter
\section*{Table of content}
\vspace{.05\textwidth}
\@starttoc{toc}
\thispagestyle{empty} % Remove page number from the ToC page
\clearpage % Start the new page without page number

\setcounter{page}{1} % Reset page counter and set page number to 1

\makeatother
\newpage

\vspace*{10pt}
\todo[inline, caption={Abstract}]{Abstract (½ page): 
    What problems is studied? 
    Why is it relevant? 
    What solutions is proposed? 
    Which contributions (theory, numerics, etc)? }
% \begin{adjustwidth}{30pt}{30pt}
%     \textit{
%         Accurate physics simulations can be computationally expensive due to the requirement of fine meshes.
%         However, a cost-effective approach involves reducing the mesh size while retaining crucial features required
%         for the simulation.
%         Some of these features are conveniently represented in the spectra of Laplacians operators defined on the mesh.
%         There exists a coarsening method that filters out unnecessary spectral bands.
%         The challenge lies in determining which spectral bands are important for the simulation, as \textbf{this information is
%         largely unknown}.
%     }\\

%     \textit{
%         To address this, we propose a \textbf{novel approach} to guide the coarsening process using \textbf{insights
%         from physical simulations}.
%         By learning the spectral subspaces that significantly influence the simulation outcomes, we can create a coarse
%         mesh that retains the characteristics essential for an accurate simulation.
%         Main contributions are:
%         \begin{itemize}
%             \item A novel spectral optimization method for coarsening, with physical simulations in mind.
%             \item Understanding importance of spectral subspaces: By understanding the spectral subspaces that contribute
%             the most to the simulations, we gain valuable insights into the underlying physics and behavior of the system.
%             \item Faster simulation times: The reduced mesh size leads to a considerable speed-up in the simulation
%             process, making it more efficient and feasible for practical applications.
%         \end{itemize}
%     }
% \end{adjustwidth}
\vspace*{10pt}

\section{Introduction (3 pages)}

\cite{altschuler2023near}

\subsection{Presentation of the problem}

\todo[inline]{Introduce the reasons to do this alg}

\nota{Let \(c\) and \(r\) respectively be the column-wise sum and the row-wise sum of matrices \textit{ie.}
\[
    c : \left\{\begin{array}{rcl}
        \M_{n,m}(\K) & \to &\K\\
        A & \mapsto &  A^\top \mathbf{1}
    \end{array}\right. \quad \text{and} \quad r: \left\{\begin{array}{rcl}
        \M_{n,m}(\K) & \to &\K\\
        A & \mapsto & A \mathbf{1}.
    \end{array}\right.
\]}

\defn{Let \(A \in \M_n(\R_+)\) be a non negative square matrix, \(\varepsilon \geq 0\) and \(k \in \N^\star\). The matrix \(A\) is \((\varepsilon, k\)\)-balanced if 
\[
    \dfrac{||c(A) - r(A)||_k}{\sum_{i,j} a_{i,j}} \leq 0.
\]
Furthermore, if \(A\) is \((0, k)\)-balanced, we say that \(A\) is balanced\todo{Use 2.3 to explain it using convex opti}.}

\defn{The \(\varepsilon, k\)-\textbf{approximate matrix balancing problem} is\todo{Use real formal definition from 2.1}: given a square non-negative matrix \(K \in \M_n(\R_+)\), \(\varepsilon \geq 0\) and \(k \in \N^\star\), find a positive diagonal matrix \(D \in \mathcal D_n(\R_+^\star)\) such that \(DKD^{-1}\) is \((\varepsilon, k)\)-balanced.}

\subsection{Related work}
Previous works (at least a few citations). If relevant, include things that you have seen during the MVA course (or possibly other courses). \\

Osborn algorithm introduced in\cite{osborne1960pre, parlett1971balancing} and defaut in Scipy.\\

\cite{chen2000balancing, chen2001preconditioning} improve accuracy of computation of eigen vectors, eigen values.\\

there are corner cases where balancing can actually worsen the conditioning.~\cite{watkins2006case} also~\cite{james2014matrix} in which they explain it and explain how to modify LAPACK to avoid it.\\

The standard matrix balancing algorithm is the Sinkhorn-Knopp algorithm~\cite{sinkhorn1967concerning}, a special case of Bregman’s balancing method~\cite{lamond1981bregman} that iterates rescaling of each row and column until convergence.
However, the algorithm converges linearly \cite{soules1991rate}, which is prohibitively slow for recently emerging large and sparse matrices.\\

\cite{parlett1971balancing} -> approximately balance matrices with a diagonal with only powers of 2. Advantage : no floating point error in computing the balanced matrix on base two computers. approximately ``good enough''\\

\cite{sugiyama2017tensor} -> matrix balancing on tensors 


\subsection{Contributions of the paper}

Their main contribution is \cref{main_thm}. It exhibits a variant of \textsc{Osborn}'s algorithm with near-linear runtime in the input sparsity. It also shows that improving the runtime dependence in \(\varepsilon\) can be improve from \(\varepsilon^{-2}\) to \(\varepsilon^{-1}\) without an additional factor \(n\).  

\thm{\label{main_thm}Let \(K \in \M_n(\R_+)\) be a balanceable non negative square matrix and \(\varepsilon \geq 0\). Random \textsc{Osborn} solves \((\varepsilon, 1)\)-approximate matrix balancing problem in \(T\) operations where there exists \(c > 0, \delta > 0\) such that 
\[
    \mathbb E(T) = \mathcal O \left(\dfrac{m}{\varepsilon}\min\left\{\dfrac{1}{\varepsilon}; d\right\} \log \kappa\right)  \quad \text{and} \quad \mathbb P \left(T \leq c \dfrac{m}{\varepsilon}\min\left\{\dfrac{1}{\varepsilon}; d\right\} \log \kappa \log \dfrac{1}{\delta}\right) \geq 1 - \delta
\]
where \(m\) is the number of nonzero entries in \(K\), \(d\) is the diameter of the graph associated to \(K\) and \(\kappa = \sum_{i,j}K_{i,j}/\min_{i,j}{K_{i,j}}\).}

\subsection{Our contributions}

numerics?
limits? 

\section{Main body (10 pages)}

\subsection{Notations}

\subsection{Presentation of the method}


\begin{code}
osborn(|\(K\)|, |\(\varepsilon\)|):
    x = 0 |\(\in \R^n\)|
    while not(is_balanced(|\(\D(e^x) K \D(e^{-x})\)|, |\(\varepsilon\)|)):
        Choose k |\(\in\)| [n] # This is where variants differ
        x += (log(c_k(|\(\D(e^x) K \D(e^{-x})\)|)) - log(r_k(|\(\D(e^x) K \D(e^{-x})\)|)))/2
    return x
\end{code}

There are \textit{many} way to choose the next coordinate to update and hence many variants of the algorithm. The article focuses on four of them:
\begin{itemize}
    \item \textbf{Cyclic \textsc{Osborn}} Cycle through the coordinates. (\textit{eg.} {\color{magenta}1, 2, 3}, {\color{cyan}1, 2, 3}, {\color{red}1,} \dots).
    \item \textbf{Random-Reshuffle Cyclic \textsc{Osborn}} Cycle through the coordinates using a new random permutation for each cycle. (\textit{eg.} {\color{magenta}2, 1, 3}, {\color{cyan}1, 2, 3}, {\color{red}1, 3, 2}, \dots).
    \item \textbf{Greedy \textsc{Osborn}} Choose \(k\) where the imbalance is maximal \textit{eg.}
    \[
        k = \argmax_{k}\left|\sqrt{r_k(\D(e^x) K \D(e^{-x}))} - \sqrt{c_k(\D(e^x) K \D(e^{-x}))}\right|.    
    \]
    \item \textbf{Random \textsc{Osborn}} Uniformly sample \(k\) independently between each call.
\end{itemize}

\todo[inline, caption={code availibility}]{Talk about the implementation}

\subsection{Theoretical guarantees}

\subsection{Numerics}

it indeed converges even with high sparsity (cf theorem)

\subsubsection{Sparsity}

We conducted numerical experiments to investigate the behavior of Osborn's algorithm in the presence of varying numbers of zero entries within randomly generated matrices. Each matrix has a size of (10, 10) with uniformly distributed values in the range [0, 1]. The number of zero entries in the matrices was systematically varied. Our objective was to assess the algorithm's ability to find solutions even with a small number of non-zero inputs, as proven in [ref].

The experiment involved measuring the execution time of Osborn's algorithm for each matrix, with the time recorded as a function of the number of zero entries. This exploration aimed to provide insights into the near-linear convergence time of the algorithm, shedding light on its performance characteristics under different sparsity levels. These results contribute valuable empirical evidence to support the theoretical findings presented in [ref]."

\todo[inline, caption={Link with convex opti}]{Use obeservation 2.5 to compare with other convex optim algorithms.}

\todo[inline, caption={Per iteration}]{Look at per iteration runtime (5.2)}

\section{Conclusion and perspective}

Summary of the result obtained: pros and cons (limitation, problems, error in the articles, etc)
Possible improvement/extension 

\section{Connexion with the course}

MANDATORY SECTION:. What are the notions/results/algorithms presented in the course that are used or related to the one presented in this paper?

\listoftodos{}

\printbibliography[heading=bibintoc]

\end{document}